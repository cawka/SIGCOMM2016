\documentclass{sig-alternate-2013}

\usepackage[english]{babel}
\usepackage[latin1]{inputenc}
\usepackage{bbm}
\usepackage{url}
\usepackage{amsmath,pifont,amssymb,amsfonts}
\usepackage{graphicx}
\usepackage{color}
\usepackage{framed}
\usepackage{gensymb}
\usepackage{textcomp}
\usepackage{longtable}
\usepackage{array}
\usepackage{subfigure}
\usepackage{psfrag}
\usepackage{algorithm}
\usepackage{algorithmic}
\usepackage{multirow}
\usepackage{balance}
\usepackage{xspace}

\newcommand{\wname}{Internet-QoE\xspace}

\makeatletter
\def\@copyrightspace{\relax}
\makeatother
\begin{document}

\title{\huge{\wname 2016}\\~\\\vspace{-5mm}\Large{ACM SIGCOMM Workshop on 
QoE-based Analysis and Management of Data Communication Networks}}

\maketitle
\section{Call for Papers}
\vspace{2mm}

Measuring and analyzing complex networks such as the Internet has been the focus 
of many research efforts for over two decades now. While this approach has lead 
to an improved understanding of the Internet and an enhancement of the 
management and operation of large-scale complex networks, little has been done 
to understand and manage the traffic and the network from a user-centric 
perspective. QoE remains a poorly understood domain, currently restricted to 
small scale lab studies and very far from the analysis of real large scale 
traffic measurements and networks. Especially in the industry, QoE has become a 
buzz word, far from its reality within the research community, and partly due to 
the complexity involved in deploying QoE-based network analysis and management 
solutions.

In addition, network operators and service providers currently struggle to keep 
their increasingly demanding customers happy in an increasingly competitive and 
complex environment, while remaining profitable at the same time. The sheer 
development of novel end-user services demands more and better user-centric 
quality concepts and metrics applied to real world operational networks. We see 
current and future networks becoming more and more end-user experience aware, 
but there is still a long way to go to make of QoE one of the guiding paradigms 
for network design, management, and operation. As a first step, we need to 
better understand real networks and their traffic through the eyes of the 
end-user.

The goal of the \wname workshop is to scale QoE out of the lab studies context 
and bring it to the design, analysis and operation of real world networks and 
traffic. By fostering an explicit and deep integration of the end-user directly 
into the analysis and management of traffic and networks, we expect to reduce 
the gap between QoE research and its application to future network management 
paradigms, as well as to provide a more targeted end-user perspective to the 
research on Internet analysis and its future development.

\wname brings together researchers and practitioners from the Internet 
measurements and analysis domain and the QoE modeling and assessment domain, as 
well as industry players willing to integrate QoE aspects into the DNA of their 
daily business, with direct applications in network dimensioning, monitoring, 
management, and troubleshooting among others. We invite submissions in the 
following non-exhaustive list of topics:

\begin{itemize}
\item QoE-aware networking\vspace{-1mm}
\item Internet measurements related to QoE\vspace{-1mm}
\item QoE characterization from Internet measurements\vspace{-1mm}
\item QoE metrics for smartphones \& mobile networks\vspace{-1mm}
\item QoE-based traffic monitoring and troubleshooting\vspace{-1mm}
\item Novel algorithms to improve Internet QoE\vspace{-1mm}
\item QoE-based network management and analysis\vspace{-1mm}
\item QoS/QoE mapping, metrics and measurements in the large-scale\vspace{-1mm}
\item Application of QoE models and metrics to network and traffic 
analysis\vspace{-1mm}
\item Tools and techniques to gather QoE-related Internet 
measurements\vspace{-1mm}
\item Large-scale field trials shedding light on QoE aspects\vspace{-1mm}
\item Qualitative measurements for Internet traffic analysis\vspace{-1mm}
\item SDN for QoE-based network management\vspace{-1mm}
\item QoE-based analysis of CDNs and Cloud networks\vspace{-1mm}
\item Large-scale network simulation for QoE analysis\vspace{-1mm}
\item Novel approaches for large-scale QoE crowd-sourcing
\end{itemize}

\subsection{Submission Instructions}
\vspace{2mm}

Submissions must be original, unpublished work, and not under consideration at 
another conference or journal. Submitted papers must be at most six (6) pages 
long, including all figures, tables, references, and appendices in two-column 
10pt ACM format. Papers must include authors names and affiliations for 
single-blind peer reviewing by the PC. Authors of accepted papers are expected 
to present their papers at the workshop. Accepted papers will be published in 
the ACM Digital Library. 

\section{Important Dates}
\vspace{2mm}

\begin{itemize}
\item Paper registration deadline: 18th March 2016\vspace{-1mm}
\item Paper submission deadline: 25th March 2016\vspace{-1mm}
\item Paper acceptance notifications: 29th April 2016\vspace{-1mm}
\item Camera ready due: 20th May 2016\vspace{-1mm}
\end{itemize}

\section{Committees}
\vspace{2mm}

\noindent \textbf{Workshop Co-chairs}

\begin{itemize}
\item Pedro Casas, Austrian Institute of Technology, Austria\vspace{-1mm}
\item Fabi�n Bustamante, Northwestern University, US\vspace{-1mm}
\item Mart�n Varela, VTT Technical Research Centre of Finland, 
Finland\vspace{-1mm}
\item David Choffnes, Northeastern University, US\vspace{-1mm}
\end{itemize}

\noindent \textbf{Program Committee}\vspace{-1mm}

\begin{itemize}
\item Rocky Chang, The Hong Kong Polytechnic University, Hong-Kong\vspace{-1mm}
\item kc Claffy, CAIDA/USCD, US\vspace{-1mm}
\item Sebastian Egger, Austrian Institute of Technology, Austria\vspace{-1mm}
\item Emir Halepovic, AT\&T Research, US\vspace{-1mm}
\item Tobias Ho�feld, University of Duisburg-Essen, Germany\vspace{-1mm}
\item Lucjan Janowski, AGH University of Science and Technology, 
Poland\vspace{-1mm}
\item Patrick Le Callet, Polytech Nantes, France\vspace{-1mm}
\item Morley Mao, University Michigan, US\vspace{-1mm}
\item Maria Papadopouli, University of Crete/ICS-FORTH, Greece\vspace{-1mm}
\item Peter Reichel, University of Vienna, Austria\vspace{-1mm}
\item Raimund Schatz, Austrian Institute of Technology, Austria\vspace{-1mm}
\item Lea Skorin-Kapov, University of Zagreb, Croatia\vspace{-1mm}
\item Shobha Venkataraman, AT\&T Research, US\vspace{-1mm}
\item Florian Wamser, University of W�rzburg, Germany\vspace{-1mm}
\item Hui Zhang, Carnegie Mellon University and CONVIVA, US\vspace{-1mm}
\end{itemize}
\end{document}